\documentclass[a4paper,11pt]{article}
\usepackage[utf8]{inputenc} % Sonderzeichen schreiben können
\usepackage[ngerman]{babel} % Deutsche Abbildungszeichen
\usepackage{graphicx}   % Graphiken einfügen
\usepackage{fancyhdr}   % Für Kopf- und Fusszeile
\usepackage{geometry}   % Seitenränder definieren
\usepackage{tikz}       % Zum Regler zeichnen
\usepackage{multirow}   % Multirow in Tabellen
\usepackage{float}      % Um Bilder genau an diesem Punkt einzufuegen [H]
\usepackage{fancyvrb}   % Box um Verbatim Code
\usepackage{microtype}  % Automatische Anpassung der Zeilenmaße, dass unschöne Umbrüche vermieden werden
\usepackage{mathtools}  % Für Text auf/unter Pfeilen
\usepackage{pgfplots}   % Plot zusammen mit tikzpicture
\usepackage{makecell}   % Umbruch in tabellen
\usepackage{enumitem}   % Einruecken bei itemize verhindern
\usepackage{titlesec}   % Elemente unter subsubsection erzeugen
\usepackage{trfsigns}   % Transformationssymbole
\usepackage{amsmath}    % Umbruch in Formeln
\usepackage{listings}   % Code darstellen
\usepackage{xcolor}     % Code farbig machen
\usepackage{longtable}  % Tabellen mit Seitenumbruch
\usepackage{textcomp}
\usepackage[font=footnotesize, labelfont=bf]{caption}
\usetikzlibrary{shapes,arrows}
\pgfplotsset{compat = newest}

\usepackage[colorlinks,
pdfpagelabels,
pdfstartview = FitH,
bookmarksopen = true,
bookmarksnumbered = true,
linkcolor = black,
plainpages = false,
hypertexnames = false,
citecolor = black] {hyperref}

% Seitenränder definieren
\geometry{
    left=25mm,
    right=25mm,
    top=25mm,
    %bottom=25mm
    }
    
\definecolor{codegreen}{rgb}{0,0.6,0}
\definecolor{codegray}{rgb}{0.5,0.5,0.5}
\definecolor{codepurple}{rgb}{0.58,0,0.82}
\definecolor{backcolour}{rgb}{0.95,0.95,0.92}

\lstdefinestyle{mystyle}{
    backgroundcolor=\color{backcolour},   
    commentstyle=\color{codegreen},
    keywordstyle=\color{magenta},
    numberstyle=\tiny\color{codegray},
    stringstyle=\color{codepurple},
    basicstyle=\ttfamily\footnotesize,
    breakatwhitespace=false,         
    breaklines=true,                 
    captionpos=b,                    
    keepspaces=true,                 
    numbers=left,                    
    numbersep=5pt,                  
    showspaces=false,                
    showstringspaces=false,
    showtabs=false,                  
    tabsize=2
}

\lstset{style=mystyle}

\titleclass{\subsubsubsection}{straight}[\subsection]

\newcounter{subsubsubsection}[subsubsection]
\renewcommand\thesubsubsubsection{\thesubsubsection.\arabic{subsubsubsection}}
\renewcommand\theparagraph{\thesubsubsubsection.\arabic{paragraph}} % optional; useful if paragraphs are to be numbered

\titleformat{\subsubsubsection}
  {\normalfont\normalsize\bfseries}{\thesubsubsubsection}{1em}{}
\titlespacing*{\subsubsubsection}
{0pt}{3.25ex plus 1ex minus .2ex}{1.5ex plus .2ex}

\makeatletter
\renewcommand\paragraph{\@startsection{paragraph}{5}{\z@}%
  {3.25ex \@plus1ex \@minus.2ex}%
  {-1em}%
  {\normalfont\normalsize\bfseries}}
\renewcommand\subparagraph{\@startsection{subparagraph}{6}{\parindent}%
  {3.25ex \@plus1ex \@minus .2ex}%
  {-1em}%
  {\normalfont\normalsize\bfseries}}
\def\toclevel@subsubsubsection{4}
\def\toclevel@paragraph{5}
\def\toclevel@paragraph{6}
\def\l@subsubsubsection{\@dottedtocline{4}{7em}{4em}}
\def\l@paragraph{\@dottedtocline{5}{10em}{5em}}
\def\l@subparagraph{\@dottedtocline{6}{14em}{6em}}
\makeatother

\setcounter{secnumdepth}{4}
\setcounter{tocdepth}{4}

\newenvironment{nospaceflalign*}
 {\setlength{\abovedisplayskip}{0pt}\setlength{\belowdisplayskip}{0pt}%
  \csname flalign*\endcsname}
 {\csname endflalign*\endcsname\ignorespacesafterend}

\pagestyle{fancy}
\fancyhf{}
\setlength{\headheight}{28.1pt} % Höhe der Kopfzeile
\rhead{\includegraphics[height=24pt]{HAWLogo}}    % HAW Logo in Kopfzeile einfügen
\lhead{Bussysteme und Sensorik}  % Beschreibung in Kopfzeile

\renewcommand{\footrulewidth}{0.4pt}    % Linie in Fußzeile einfügen
\rfoot{\thepage}    % Seitenzahlen einfügen
\lfoot{Department Informations- und Elektrotechnik}   % Fusszeile linksbündig

\begin{document}
%Titelseite
\begin{titlepage}

  \begin{figure}
    \centering
    \includegraphics[height=2.5cm]{HAWLogo}
  \end{figure}

  \vspace*{2cm}
  \centering
  {\scshape\Large Bussysteme und Sensorik \par}
  \vspace{1cm}
  {\scshape\LARGE Entwicklung eines Hubs zur Erfassung und graphischen Darstellung von Sensordaten \par}
  \vspace{0.5cm}
  {\scshape\large Wintersemester 2023/2024 \par}
  \vspace{5cm}

  \raggedright
  Ausarbeitung von:

  \vspace{0.5cm}
  Lasse Kelling \\
  123456 % matnr

  \vspace{0.2cm}
  Fabian Schmalenbach \\
  2514071

  \vspace{0.5cm}
  Abgabedatum: 24.06.2023

  \vspace{0.5cm}
  Prüfer: Prof. Dr. R. Fitz



\end{titlepage}

\newpage
\addtocontents{toc}{\protect\thispagestyle{empty}}
\tableofcontents
\thispagestyle{empty}
\newpage

\setcounter{page}{1}    % Seitenzähler sicherheitshalber zurücksetzen

\section{Projektbeschreibung}
\label{sub:projektbeschreibung}

Ziel des Projekts ist die Entwicklung eines Sensorhubs, der Sensordaten erfasst und auf einem Display darstellt.
Bei den Sensoren handelt es sich in erster Linie um Umweltdaten, die aktuelle Parameter der Umgebung erfassen.
Das Projekt kann daher grob mit einer Wetterstation verglichen werden.

\subsection{Anforderungen}
\label{subsub:anforderungen}

Es wurden keine verpflichtenden Anforderungen gestellt, das Projekt soll thematisch aber zum Modul
"Bussysteme und Sensorik" passen. Daraus lassen sich für das spezifische Projekt Anforderungen stellen bzw. ableiten:

\begin{itemize}
  \item Verwendung eines oder mehrerer Bussysteme zur Kommunikation zwischen Mikrocontrollern
  \item Nutzung diverser Sensoren mit unterschiedlichen Anbindungen für Vielfältigkeit
  \item Analog zu herkömmlichen Wetterstationen, soll diese ebenfalls über einen Außensensor verfügen
  \item Die Wetterstation soll über eine Wettervorhersage verfügen
  \item Der Sensorhub soll skalier- und erweiterbar sein
\end{itemize}

\noindent
Aus diesen Anforderungen lassen sich direkt Vorgaben für das Projekt ableiten:
\begin{itemize}
  \item Nutzung mehrerer Mikrocontroller, die miteinander über ein Bussystem kommunizieren
  \item Verwendung digitaler Sensoren, die Standardprotokolle wie I2C, SPI oder UART unterstützen
  \item Entwicklung eines Außensensors, der drahtlos mit dem Sensorhub kommunizieren kann
  \item Anbindung des Sensorhubs ans Internet oder Empfang von Wettervorhersagen via Funk (DCF77)
  \item Nutzung ausreichend leistungsstarker Mikrocontroller, die genügend Leistungs- und Peripheriereserven haben, um weitere Geräte anzubinden
\end{itemize}

\section{Systembeschreibung}
\label{sub:systembeschreibung}

\subsection{Systemaufbau}
\label{subsub:systemaufbau}

\begin{figure}[H]
  \centering
  \includegraphics[width = 0.8\textwidth]{Systemuebersicht}
  \caption{Übersicht der Systemkomponenten}
  \label{fig:systemuebersicht}
\end{figure}

Die obige Abbildung \ref{fig:systemuebersicht} zeigt abstrakt alle Komponenten des Systems. Die jeweiligen Komponenten über dem Mikrocontroller Symbol
zeigen jeweils die angeschlossenen Sensoren bzw. Bildschirme. Ebenfalls ist grob die Kommunikation zwischen den Mikrocontrollern erkennbar. 

\vspace{0.2cm}
\noindent
Ganz links ist der Außensensor dargestellt, an den ein Sensor zur Messung von Temperatur, Luftfeuchtigkeit und Luftdruck angeschlossen ist. 
Außerdem verfügt dieser über eine Antenne, um das DCF77-Signal zu empfangen, welches die aktuelle Zeit und eine Wettervorhersage beinhaltet (siehe Abschnitt \ref{subsubsub:dcf77}).
Der Sensor sendet die empfangenen Daten zyklisch per 433 MHz Sender an den Sensorhub. 

\vspace{0.2cm}
\noindent
Der Sensorhub besteht aus zwei Mikrocontrollern, die über eine serielle Schnittstelle (UART) miteinander kommunizieren. Der in Abbildung \ref{fig:systemuebersicht} mittlere Mikrocontroller
empfängt die Daten des Außensensors und leitet diese weiter an den verbundenen Mikrocontroller. Da die Daten zur Wettervorhersage verschlüsselt sind und der Außensensor möglichst
wenig Energie verbrauchen soll, müssen die Daten vom Mikrocontroller entschlüsselt werden. Dazu werden die entsprechenden Datenpakete abgefangen, entschlüsselt und anschließend weitergesendet. 
Der Mikrocontroller verfügt außerdem wie der Außensensor über einen Sensor zur Messung von Temperatur, Luftfeuchtigkeit und Luftdruck (wobei der Luftdruck im Innenraum nicht gemessen wird). 
Zusätzlich ist ein CO2 Sensor verbaut, der die Konzentration im Raum misst. 

\vspace{0.2cm}
\noindent
In der Übersicht ganz rechts ist der Mikrocontroller, der alle Daten empfängt und auf einem Touchdisplay darstellt. Da der Sensor über ein WLAN-Modul verfügt, wäre theoretisch
zusätzlich die Übertragung der Daten per WLAN an einen Server o.ä. möglich, der alle Daten speichert und diese anderen Geräten zur Verfügung stellt. 

\subsection{Sensorik}
\label{subsub:sensorik}

Zur Erfassung der Umweltparameter werden aktuell zwei verschiedene Sensoren eingesetzt. Der BME280 erfasst Lufttemperatur, Luftfeuchtigkeit und Luftdruck. 
Der MHZ19C wird zur Erfassung der CO2 Konzentration im Raum eingesetzt. In den folgenden beiden Abschnitten werden die Sensoren beschrieben. 

\vspace{0.2cm}
\noindent
Das Außenmodul verfügt außerdem über eine 77,5 kHz Empfangsantenne, um das Zeitzeichensignal DCF77 zu empfangen. Daraus lässt sich die aktuelle Zeit ablesen,
außerdem wird eine Wettervorhersage mit übertragen, die ausgewertet wird. 

\subsubsection{BME280}
\label{subsubsub:bme280}

Beim BME280 handelt es sich um einen effizienten Sensor von Bosch, der zur Erfassung von Temperatur, Luftfeuchtigkeit und Luftdruck eingesetzt wird. 
Der Sensor verfügt über ein I2C und SPI Interface.
Beim zyklischen Auslesen aller Sensordaten mit 1 Hz liegt die Stromaufnahme laut Datenblatt bei 3,6 $\mu$A, weshalb sich der Sensor ideal für die Anwendung in Sensormodulen mit Batteriebetrieb eignet. 

Der Sensor verfügt über die folgenden Messbereiche:
\begin{itemize}
  \item Temperatur: -40°C - +85°C ($\pm$ 0,5°C)
  \item Luftfeuchtigkeit: 0 - 100\% rel. Feuchtigkeit ($\pm$ 3\%)
  \item Druck: 250 - 1250 hPa ($\pm$ 1 hPa)
\end{itemize}

\noindent
Mit diesen Spezifikationen eigent sich der Sensor für die Anwendung im Innen- und Außenbereich. 

\noindent
Für das Projekt wird ein Sensorshield verwendet, welches nur den I2C Bus herausführt. Die Ansteuerung des Sensors erfolgt daher über I2C. 

\subsubsection{MHZ19C}
\label{subsubsub:mhz19c}

Der MHZ-19C ist ein Infrarot CO2-Sensor, der die CO2 Konzentration mittels nicht-dispersiver Infrarot-Spektroskopie misst. Dabei wird zyklisch mit einer Infrarotlampe
und einem Photosensor die CO2-Konzentration anhand der Reflexion des Lichts durch CO2 Partikel gemessen. 
Der Sensor verfügt über einen Messbereich von 400-5000 ppm, wobei die Genauigkeit $\pm$ 40ppm + 5\% des Messwerts beträgt. 
Da NDIR Sensoren zum Messwertdrift neigen, verfügt der Sensor über eine automatische Kalibrierung, die alle 24 Stunden erfolgt. Der Sensor ist daher für den Dauerbetrieb ausgelegt
und sollte dauerhaft aktiv sein. 
Die Aufheizzeit des Sensors beträgt eine Minute. 

\noindent
Der Sensor ist über eine UART-Schnittstelle konfigurier- und auslesbar, außerdem verfügt er über einen PWM Ausgang. 

\subsubsection{DCF77}
\label{subsubsub:dcf77}

\subsubsubsection{MeteoTime}
\label{subsubsubsub:meteotime}

\subsection{Kommunikation}
\label{subsub:kommunikation}

\subsubsection{Funkstrecke}
\label{subsubsub:funkstrecke}

\subsubsection{Datenpakete}
\label{subsubsub:datenpakete}

\section{Modulbeschreibung}
\label{sub:modulbeschreibung}

\subsection{Sensormodul extern}
\label{subsub:sensorModul_ext}

\subsection{Sensormodul intern}
\label{subsub:sensorModul_int}

\subsection{Grafikmodul}
\label{subsub:grafikmodul}

\section{Aktueller Projektstand}

\end{document}
